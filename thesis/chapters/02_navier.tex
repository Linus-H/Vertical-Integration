\chapter{Navier Stokes Equations}\label{chapter:navier_stokes}
As a case study in this thesis the vertical part of the Navier Stokes Equations with some simplifications will be used.
The following variables are used in the equations (TODO: quote introduction to numerical weather prediciton):
\begin{itemize}
\item $t$ time (in $s$)
\item $\textbf{V}_3$ wind speed (in $\frac{m}{s}$)
\item $T$ temperature (in $K$)
\item $\rho$ density (in $\frac{kg}{m^3}$)
\item $q$ specific humidity (in $\frac{g}{kg}$)
\item $p$ pressure (in $\frac{N}{m^2}$)
\item $\boldsymbol{\Omega}$ angular velocity of the Earth (in $\frac{m}{s}$)
\item $\Phi = gz$ geopotential comprising effects of gravity and centrifugal force (in $\frac{J}{kg}$)
\item $R=8.314\frac{J}{mol\cdot K} = 287.0 \frac{J}{kg\cdot K}$ perfect gas constant
\item $C_p=28.97 \frac{J}{mol\cdot K} = 1000 \frac{J}{kg\cdot K}$ specific heat at constant pressure for dry air (TODO: source for numbers)
\item $\textbf{F}$ source/sink term for momentum (in $\frac{m}{s^2}$)
\item $\textbf{Q}$ source/sink term for heat (in $\frac{J}{mol\cdot K}$)
\item $\textbf{M}$ source/sink term for specific humidity (in $\frac{g}{s\cdot kg}$)
\end{itemize}
Using the material derivative $\frac{D\bullet}{Dt}=\frac{\partial\bullet}{\partial t}+\textbf{V}_3\cdot \nabla_3\bullet$ the most general form of the equations can be written follows:
\begin{align}
\text{momentum equation}\;\; \frac{D\textbf{V}_3}{Dt} &= -2\boldsymbol{\Omega}\times \textbf{V}_3 - \frac{1}{\rho}\nabla _3 p - \nabla _3 \Phi + \textbf{F} \label{eq_mom}\\
\text{thermodynamic equation}\;\;\; \frac{DT}{Dt} &= \frac{R}{C_p}\frac{T}{p}\frac{dp}{dt}+\frac{\textbf{Q}}{C_p}\label{eq_therm}\\
\text{continuity equation}\;\;\; \frac{D\rho}{Dt} &= -\rho \nabla _3 \cdot \textbf{V}_3\label{eq_cont}\\
\text{water vapor equation}\;\;\; \frac{Dq}{Dt} &= \textbf{M}\label{eq_water}\\
\text{equation of state}\;\;\;\;\; p &= \rho R T
\end{align}
\subsubsection{Interpretation of the Terms:}
The material derivative accounts for the fact that the physical properties of a fluid parcel need to 'travel' with their fluid parcel through space.
Starting with the momentum equation \ref{eq_mom}:
\begin{itemize}
\item $-2\boldsymbol{\Omega}\times \textbf{V}_3$ this term describes the Coriolis effect, which needs to be modeled because the rotating earth is used as the frame of reference.
\item $- \frac{1}{\rho}\nabla _3 p$ this term describes how air flows from high to low pressure.
\item $- \nabla _3 \Phi$ this term describes how air flows from high potential to low potential, i.e. in the direction of gravity.
\end{itemize}
For complete accuracy a term of the shape $\frac{\mu}{\rho} \nabla _3^2 \textbf{V}_3$ accounting for the viscosity can be added.
However $\mu \approx 1.7\cdot 10^{-5}$, so the term can often be ignored.

In the thermodynamic equation \ref{eq_therm}, the term $\frac{R}{C_p}\frac{T}{p}\frac{dp}{dt}$ describes how temperature changes given how the pressure changes.
This relationship depends on temperature and pressure, because the thermodynamic properties of air also depend on the current temperature and pressure.
To be completely accurate a term of the shape $\frac{\alpha R}{C_p}\nabla _3^2 T$ should be added, to account for thermal conduction, but for these equations they can be neglected, because $\alpha \approx \frac{0.024}{R\rho}$ (TODO: source) is sufficiently small.

The continuity equation \ref{eq_cont} can also be written as:
\begin{align*}
\frac{\partial \rho}{\partial t} =  - \textbf{V}_3 \cdot \nabla _3 \rho -\rho \nabla _3 \cdot \textbf{V}_3 
\end{align*}
The first term accounts for how density travels along with the flow.
The second term describes how the density of a gas is reduced if the flow is divergent, i.e. if more material flows away from a point in space than flows in, the amount of material in that point reduces.\\
In the following the source/sink terms $\textbf{F}$, $\textbf{Q}$ and $\textbf{M}$ are set to zero, and the water vapor equation \ref{eq_water} is ignored.


\section{Assumptions, Simplifications, and Modifications}
In this section the most common simplifications will be listed and explained.

\subsection{Splitting into Vertical and Horizontal}
Commonly the Navier Stokes equations are split up into its vertical and horizontal components.
To this end two new variables $\textbf{V} \in \mathbb{R}^2$ and $w\in \mathbb{R}$, representing horizontal and vertical winds respectively, are introduced.
Additionally $\boldsymbol{k}$ is the vertical vector (always perpendicular to the ground), and $f=2\Omega \sin \phi$ is the Coriolis parameter.
Assuming gravity to only act in the vertical and being constant across the atmosphere, we can write $-\nabla \Phi = -g \boldsymbol{k}$.
Using these definitions to replace $\textbf{V}_3$, we end up with the following equation system:
\begin{align*}
\frac{D\textbf{V}}{Dt} &= -f\boldsymbol{k} \times \textbf{V} - \frac{RT}{p}\nabla p\\
\frac{D\boldsymbol{k}}{Dt} &= \frac{RT}{p} \frac{\partial p}{\partial z} - g \\
\frac{DT}{Dt} &= \frac{R}{C_p}\frac{T}{p}\frac{dp}{dt}\\
\frac{Dp}{Dt} &= -\frac{p}{1- \frac{R}{C_p}} (\nabla \textbf{V} + \frac{dw}{dz})\\
p &= \rho R T
\end{align*}
\subsection{Ignoring Horizontal Winds}
\subsection{Replacing $p$ by $\text{ln}p$}
\subsection{Assuming Constant Temperature}
\subsection{Hydrostatic Assumption}
give all simplifications made to the navier stokes equations names for later chapters

\subsection{Alternative Coordinate Systems}
general explanation of alternate vertical coordinate systems\\
introducing pressure-coordinates and explaining their advantages
\subsection{Boundary Conditions}
discussion of how each of the state variables should behave at the boundaries of the system
\begin{itemize}
\item winds are zero at upper and lower boundary
\item density is zero at upper boundary
\item integral over density in vertical needs to be constant over time. (preservation of mass)
\item Temperature ?
\item pressure is zero at upper boundary
\item pressure and density at lower boundary ?
\end{itemize}

