\chapter{Numerical Analysis}
\section{Validation}
In this section it will be explained how all of the above theory has been validated.
Explain how the individual functions were unit-tested.
\subsection{Differential Operators}
\subsection{Integrators}
\subsection{Implementation of Differential Equations}

\section{Analysis of Errors}\label{chapter:introduction}
In this section the effects (all?) possible choices of simplifications, integrators etc. is analyzed.\\
To this end we introduce different ways of measuring errors, i.e. different norms, and use them to measure the difference between different simulations.

% analysis of errors
\subsection{Differential Operators}
show how the error-order is influenced by grid-size.\\
grid size far larger than features to be measured => inaccuracy\\
grid size a little bit smaller than features to be measured => change in accuracy according to error-order of implemented operator\\
grid size too small for machine precision => loss of accuracy

\subsection{Integrators}
\subsubsection{Runge-Kutta-Integrators}
explain how time-step-size and grid-size have to change together, and show trough examples how accuracy changes depending on the choice of these two parameters (maybe a heatmap with x-axis = time-step-size, y-axis=grid-size, color=accuracy after simulation time T?)
\subsubsection{Exponential Integrators}
showcase how accuracy stays constant independent of time-step-size (for linear systems)\\
give a small example of simulating a non-linear system by linearization around the current state

\subsection{Simplifications to the Navier Stokes Equations} % effects of simplifications
have one simulation without simplifications run with high precision (RK4 with tiny step-sizes and high grid resolution), and compare that to simulations using high precision with simplifications.

\subsection{Grid Discretizations}% grid discretizations
Quantify the difference between the two different grids.
How do we really know which one is better?
Find ways to quantify which one is more accurate?