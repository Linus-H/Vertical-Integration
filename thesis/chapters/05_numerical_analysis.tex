\chapter{Numerical Studies}
Having discussed how the NSE were simplified, discretized, and implemented, in this section the implementation is tested and validated.
To this end, first the results of the unit tests described in section \ref{sec:testing} are discussed, by looking at the numerical errors made by the differential operators and integrators.
Thereafter the implementations of the non-hydrostatic NSE discussed in section \ref{sec:non_hydrostatic} are analyzed.

\section{Validation by Analyzing Numerical Errors}
In this section the results of the tests performed in section \ref{sec:testing} are discussed.

% analysis of errors
\subsection{Differential Operators}
Test-Function: $\sin(20\pi (x-e))$ on domain $[0;1]$, with resolutions from 5 to $20\cdot 10^7 $
Actual result is $20\pi\cos(20\pi (x-e))$\\
show how the error-order is influenced by grid-size.\\
grid size far larger than features to be measured => inaccuracy\\
grid size a little bit smaller than features to be measured => change in accuracy according to error-order of implemented operator\\
grid size too small for machine precision => loss of accuracy

\subsection{Integrators}
\subsubsection{Runge-Kutta-Integrators}
First off, to verify the implementation, the RK-Integrators were used to solve a differential equation without any spacial resolution, but with temporal variance.
$\frac{dx}{dt}=\frac{t}{t^2+1}$
with solution $x_0 + 0.5\text{ln}(t^2+1)$.

explain how time-step-size and grid-size have to change together, and show trough examples how accuracy changes depending on the choice of these two parameters (maybe a heatmap with x-axis = time-step-size, y-axis=grid-size, color=accuracy after simulation time T?)
\subsubsection{Exponential Integrators}
showcase how accuracy stays constant independent of time-step-size (for linear systems)\\
give a small example of simulating a non-linear system by linearization around the current state


\section{Analysis of Errors}
In this section the effects (all?) possible choices of simplifications, integrators etc. is analyzed.\\
To this end we introduce different ways of measuring errors, i.e. different norms, and use them to measure the difference between different simulations.

\subsection{Simplifications to the Navier Stokes Equations} % effects of simplifications
have one simulation without simplifications run with high precision (RK4 with tiny step-sizes and high grid resolution), and compare that to simulations using high precision with simplifications.

\subsection{Grid Discretizations}% grid discretizations
Quantify the difference between the two different grids.
How do we really know which one is better?
Find ways to quantify which one is more accurate?