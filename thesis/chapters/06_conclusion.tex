% !TeX root = ../main.tex
% Add the above to each chapter to make compiling the PDF easier in some editors.

\chapter{Conclusion}\label{chapter:conclusion}
In this chapter the thesis is summarized and some possible areas of future work based on the Python tool are suggested.
The code is freely available for this purpose as a Github repository at \url{https://github.com/Linus-H/Vertical-Integration}.
In this chapter the work is be summarized and some further possible future work is be discussed.

\section{Summary}
In this thesis, first, a basic introduction to the Navier Stokes Equations for numerical weather prediction was given.
Having defined the most general equation system, this system was then rearranged and simplified, by assuming that horizontal wind $\textbf{V}$ is zero.
Thereafter, the variable $z$ was substituted by a new vertical coordinate $s$, which leads to a new equation system with some advantageous properties, such as a high spatial resolution in the lower areas of the atmosphere in contrast to the low spatial resolution at the top of the atmosphere.
On top of this, this substitution allowed for the upper bound for the domain, to be found dynamically instead of setting it at the beginning of the simulation manually.\\
This was the equation system that was used in the later sections.\\
Having established the version of the NSE to be simulated, the applied methodology for spatial and temporal discretization was explained in section \ref{chapter:discretization}.\\
Then the modular framework of the software was outlined in section \ref{chapter:implementation}, and an example of an implementation was given at the end of that section.\\
Finally the tests conducted on the implementation and their results were described in section \ref{chapter:numerical_study}.


\section{Future Work}
There are multiple classes for expanding the work done in this thesis.\\
The first category consists of more experiments that could be conducted in the non-hydrostatic equation-system.
\begin{itemize}
\item Observing the effects of varying temporal and spatial resolution; the effects would be expected to be quite significant, as the non-hydrostatic equation system is non-linear in nature.
\item Experimenting with other time integrations methods/other integrators, to see whether they yield significantly different results from the Runge-Kutta-methods primarily used in this thesis.
\item Measuring the wave-speeds/speed of sound of waves at different wave-lengths. 
\end{itemize}
The second category of possible expansion are comparative in nature:
\begin{itemize}
\item compare the simulation results of the non-hydrostatic NSE to simulation results from implementations using the hydrostatic NSE
\item compare the non-hydrostatic implementation to real-world measurements, by adding the remaining two dimensions of the horizontal to the implementation.
\end{itemize}
Lastly, it would be interesting to introduce standardized benchmarks for the non-hydrostatic NSE.
So far, such benchmarks only exist for systems using the hydrostatic NSE (TODO Source for hydrostatic Benchmarks), making standardized comparisons between different implementations of then non-hydrostatic NSE impossible.