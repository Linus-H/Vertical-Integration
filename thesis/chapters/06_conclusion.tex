% !TeX root = ../main.tex
% Add the above to each chapter to make compiling the PDF easier in some editors.

\chapter{Conclusion}\label{chapter:conclusion}
In this chapter we summarize the thesis and suggest some possible areas of future work based on the Python tool are suggested.
The code is freely available for this purpose as a Github repository at \url{https://github.com/Linus-H/Vertical-Integration}.

\section{Summary}
In this thesis we first gave a basic introduction to the Navier Stokes Equations for numerical weather prediction.
Having defined the most general equation system, we then rearranged and simplified this system, by assuming that horizontal wind $\textbf{V}$ is zero.
Thereafter, we substituted the variable $z$ by a new vertical coordinate $s$ which leads to a new equation system with some advantageous properties.
For example, the substitution leads to a high spatial resolution in the lower areas of the atmosphere in contrast to the low spatial resolution at the top of the atmosphere.
On top of this, this substitution allows for the upper bound for the domain to be found dynamically instead of setting it at the beginning of the simulation manually.
We employed the resulting equation system in the later sections.

Having established the version of the NSE to be simulated in Chapter~\ref{chapter:navier_stokes}, we explained the applied methodology for spatial and temporal discretization in Chapter~\ref{chapter:discretization}.
Then we outlined the modular framework of the software in Chapter~\ref{chapter:implementation}, and gave an example of an implementation at the end of the chapter.
Finally, we described the tests conducted on the implementation and their results in Chapter~\ref{chapter:numerical_study}.


\section{Future Work}
There are multiple directions in which the the work done in this thesis can be expanded.
The first category consists of more experiments that could be conducted on the non-hydrostatic equation-system.
\begin{itemize}
\item Observing the effects of varying temporal and spatial resolution; the effects would be expected to be quite significant, as the non-hydrostatic equation system is non-linear.
\item Experimenting with other time integrations methods/other integrators, to see whether they yield significantly different results from the Runge-Kutta-methods primarily used in this thesis.
\item Measuring the wave-speeds/speed of sound of waves at different wave-lengths. 
\end{itemize}
The second category of possible expansion are comparative in nature:
\begin{itemize}
\item compare the simulation results of the non-hydrostatic NSE to simulation results from implementations with the hydrostatic NSE
\item compare the non-hydrostatic implementation to real-world measurements, by adding the remaining two dimensions of the horizontal to the implementation.
\end{itemize}
Lastly, it would be interesting to define standardized benchmarks for the non-hydrostatic NSE.
So far, such benchmarks are mainly adopted for systems employing the hydrostatic NSE~\cite{williamson1992standard}, making standardized comparisons between different implementations of then non-hydrostatic NSE difficult.